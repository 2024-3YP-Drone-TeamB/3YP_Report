\newpage
\fancyhead[C]{Thomas Turner}

\section{Conclusion} \label{conclusion}

The work put forward is in line with our design and mission objectives. In this project we have shown the viability of the design at many levels. Sections \ref{section:Custom Hardware} and \ref{sec:bms} demonstrate how custom hardware will be used in order to meet the specific needs of the harsh operating conditions in line with the design goals and at lower cost than \gls{COTS} options. In Section \ref{sensor_hardware_data_acquisition} it is demonstrated how \gls{COTS} options can be used to gather high quality data needed for advanced data analysis specific to the task of removing anti-personal mines.

Section \ref{computervision} shows how the data acquired from the selected sensors can be used to detect anti-personal mines using simulations built with real world data from the region of interest using sensor fusion of \gls{GPR} and thermal data. Section \ref{sec:msp} demonstrates how missions will be planned to avoid obstacles and allow multi-agent drones to operate at peak efficiency using the innovative divide-and-cluster technique. 

Sections \ref{sec:euc} and \ref{Intra Communication} address how the device will communicate with the base station and between modules within the device considering modularity and redundancy at the forefront. The control system of the drone is also considered in Section \ref{sec:simcontrol} in order to optimise the stability and performance of the drone considering and considering multiple agents. 

In Section \ref{Return to Safety} a novel Return to Safety procedure is considered using adaptive control and path planning to attempt to land in safe regions if experiencing faults. Finally, project financing, sustainability, safety and risk, technology strategy are considered to ensure the long-term health of the project. 

In summary, from hardware to sensors to computer vision and mission planning all stages are considered to ensure the project is viable and effective.