\newpage
\fancyhead[C]{Thomas Turner}

\section{Conclusion} \label{conclusion}

The work put forward is in line with our design and mission objectives outlined in Section \ref{introduction}. In this project, we have shown the viability of the design at many levels. Sections \ref{section:Custom Hardware} and \ref{sec:bms} demonstrate how custom hardware will be used to meet the specific needs of the harsh operating conditions in line with the design goals and at a lower cost than \gls{COTS} options. Section \ref{sensor_hardware_data_acquisition} demonstrates how \gls{COTS} options can be used to gather high-quality data needed for advanced data analysis specific to removing anti-personnel mines.

Sections \ref{computervision} and \ref{fusion} show how the data acquired from the selected sensors can be used to detect anti-personnel mines, validated with simulations of real-world conditions and data from the regions of interest using sensor fusion of \gls{GPR} and thermal data. Section \ref{sec:msp} demonstrates how missions will be planned to avoid obstacles and allow multi-agent drones to operate at peak efficiency using the innovative divide-and-cluster technique. 

Sections \ref{sec:euc} and \ref{Intra Communication} address how the device will communicate with the base station and between modules within the device, considering modularity and redundancy at the forefront. The control system of the drone is also considered in Section \ref{sec:simcontrol} to optimise the stability and performance of the drone, considering multiple agents. 

In Section \ref{Return to Safety}, a Return to Safety procedure is considered using adaptive control and path planning to attempt to land in safe regions if experiencing faults. Finally, project financing, sustainability and ethics, safety and risk, and technology strategy are all considered to ensure the long-term health of the project. 

Sections \ref{Safety and Risk}, \ref{financing}, \ref{sec:techstrat}, and \ref{sustainability_ethics} investigate the safety and risk, project financing, technology strategy, and ethical and sustainability aspects of the project, respectively.

%In summary, from hardware to sensors to computer vision and mission planning, all stages are considered to ensure the project is viable and effective. This means the project is ready for the testing and validation stages where the models and parameters can be tested and optimised. Furthermore, real world data can be gathered to train and improve computer vision, path planning and device control.

In summary, all stages of the system—from hardware and sensors to computer vision and mission planning—have been carefully considered to ensure technical viability and operational effectiveness. The project is now ready for the testing and validation phase, during which real-world data can be collected to refine models, optimise system parameters, and benchmark the overall system performance in operational conditions.