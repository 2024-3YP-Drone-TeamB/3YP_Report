\subsection{Fault Detection, Quantification and Control}\label{sub_section:tgt_fault_detection}

\subsubsection{State of Charge}\label{sub_sub_section:tgt_SOC}
\paragraph{Component Ageing}
Cell ageing and actuator ageing can cause inaccurate state of charge predictions. Therefore, the planned path may exceed the limit of the device, leading to a state of charge fault and the necessity of \gls{RTH} if the state of charge is within 10\% of the predicted required state of charge required for \gls{RTH}. However, the state-vector telemetry recorded by the flight controller will be used to regularly inspect the cell performance and replace the cells or actuators if needed before failure.
\paragraph{Cell Failure}
\gls{LIION} batteries can have a thermal runaway. This is seen with a massive spike in temperature from the \gls{BMS} telemetry data and should lead to immediate landing and battery shutdown as it is a chain reaction effect that can cause excessive thermal damage\cite{LiONRunaway}. To avoid this, if the battery temperature exceeds the safe bounds designated by the \gls{BMS}, the device will land and execute \gls{RTS} when the battery has cooled sufficiently.

\subsubsection{Actuator Fault}\label{sub_sub_section:tgt_actuator_fault}
\paragraph{Causes}
A loss of actuator effectiveness can be either from motor damage or propeller damage. The possible causes of motor damage specific to the operating environments include: wire damage, cooling blockages and dust or sand getting into the components. For propeller damage this could be from impacts, thermal effects or fatigue cracks.
\paragraph{Monitoring}
Consider motor effectiveness factors $\eta_i \in [0,1]$ as random walks. Using the \gls{RPM} telemetry, predicted, thrust for each motor $T_{\text{pred},i} = \eta_i k_T \omega_i^2$ can be generated. This is used to predict the acceleration using an \gls{EKF} observer, $\mathbf{a}_{prediction}$, of the device and then compare with IMU-measured acceleration,  $\mathbf{a}_{actual}$, to calculate the residual as shown in Equation \ref{eq:residual}. 
\begin{equation}\label{eq:residual}
    \mathbf{r}(t) = \mathbf{a}_{prediction}(t) - \mathbf{a}_{actual}(t)
\end{equation}
The model parameters $\eta_i$ can be approximated using \gls{EKF} updates. An \gls{EKF} is used due to its wide industry use and proven effectiveness, and should be sufficient to handle the non-linearities of the model; however, if it is found to be insufficient, a \gls{UKF} filter could be used instead \cite{WAN2000}.

\paragraph{Fault Detection}
If $\eta_i$ falls below $\eta_{\text{thresh}}$, failure is detected; however, this has a slow detection time for sudden faults. Given the case of sudden faults, \gls{RTS} should not be attempted due to the likelihood of crashing; a separate detection system is required to trigger emergency landing. The residual $\mathbf{r}(t)$ is shown in Equation \ref{eq:residual}. If there is a sudden spike in residual, it is likely due to an extreme gust, a change in state dynamics, or a sudden fault. In each case emergency landing should be attempted immediately. To detect this spike with changing residual dynamics which can be caused by increased wind or sensor drift, the mean and covariance of $\mathbf{r(t)}$ are found as shown in Equations \ref{eq:residual_mean} and \ref{eq:residual_cov} where  $\mathbf{\tilde{r}}(t) = \boldsymbol{\tilde{r}}(t) - \boldsymbol{\mu_r}(t)$, and $\alpha \in [0,1)$ is the forgetting factor that will be tuned during testing \cite{roberts1959}.
\begin{equation}\label{eq:residual_mean}
     \boldsymbol{\mu_r}(t) = \alpha \boldsymbol{\mu_r}(t-1) + (1 - \alpha) \mathbf{r}(t)
\end{equation}
\begin{equation}\label{eq:residual_cov}
    \boldsymbol{\Sigma_r}(t) = \alpha \boldsymbol{\Sigma_r}(t-1) + (1 - \alpha) \boldsymbol{\tilde{r}}(t)\boldsymbol{\tilde{r}}(t)^T
\end{equation}
The acceptance range is then set using Equation \ref{eq:bounds} with sensitivity $k$ that can be tuned during testing. Then, if $\mathbf{r(t)}$ has a value outside the acceptable region, a fault is triggered \cite{Perry2010}. This provides a computationally lightweight algorithm well suited for real-time embedded applications that automatically adjusts the detection bounds based on the characteristics of the sensors. This is especially important in \gls{UAV}s due to the gyroscopic drift and exposure to gusty winds that can cause false positives with fixed bounds.
\begin{equation}\label{eq:bounds}
    C_{\text{acceptance}}(t) = \boldsymbol{\mu_r}(t) \pm k \sqrt{\text{diag}(\boldsymbol{\Sigma_r}(t))}
\end{equation}

\paragraph{Calibration}
The fault detection and estimation methods used are
dependent on accurate estimation of plant and sensor characteristics. Therefore, regular calibration is required by running boot tests that run fixed control inputs and record sensor outputs. If the sensor outputs do not match the predicted values, the device raises an error and is debugged at the base station. The \gls{LED} array explored in Section \ref{section:Custom Hardware} displays the faulty module using sensor isolation where possible. For example, if all inertial sensors match the predicted value except one, this sensor would be isolated as faulty.

\paragraph{Reconfiguration}
For path planning as discussed in Section \ref{sub_sub_section:tgt_weather_failure}, we need known characteristics of the device, specifically its travel speed and maximum wind speed rejection. Therefore, pre-computed control gains are calculated for different intervals of $\eta$, simulated, and then tested with hardware-in-the-loop using fault injections. These controllers are preloaded into the device. These controllers are more robust to parameters and less aggressive than during typical operation to minimise the risk of actuator saturation, instability due to parameter inaccuracy, and further damage to actuators. For the control behaviour to remain applicable with differing model parameters the motor control signals, $\mathbf{E}$, are scaled as shown in Equation \ref{eq:adj_thrust} where $\mathbf{N} = \text{diag}(\eta_1, \dots,\eta_4)$.
\begin{equation}\label{eq:adj_thrust}
    \mathbf{E}_{\text{adjusted}} = \mathbf{N}^{-1}\times \mathbf{E}_{\text{control}}, 
\end{equation}
If any $\eta_i$ falls below a threshold $\eta_{\text{thresh}}$ for each controller, it will be transferred to different control gains, and if any $\eta_i$ falls below $\eta_{\text{critical}}$ it will trigger emergency landing. To ensure transitions between controllers are smooth, the transition step should be simulated and tested with hardware before deployment. If this proves ineffective due to saturation of the damaged actuators, constrained quadratic programming can be used \cite{JOHANSEN2013} to distribute thrust to the stronger motors.

\begin{comment}
Furthermore, using static thresholds can lead to false positives when performing aggressive manoeuvres or in gusty conditions. This can be mitigated using statistical methods as explored in \cite{REF}, but given that the device follows smooth paths and does not fly in extreme weather, this mitigation is unnecessary.
\paragraph{Control Strategy}
The simplest solution is to tune the output signals from the controller until the thrust of the actuator matches the expected value. This allows the control loop to be unchanged, abstracting from the physical effects. However, the actuator will saturate at a lower thrust value, meaning the control loop will not perform as expected and can lead to failure in typically non-failure states. By increasing the output on the actuator, it will likely cause the current defect to degenerate as the loads increase on the damaged actuator. Therefore, a gain scheduling approach is used with a selection of less aggressive controller gains to match different tuning magnitudes. As the tuning magnitude increases, the controller should be less aggressive so that the thrust demands reduce. However, this means that the controller cannot tolerate disturbances of the same magnitude. 

    \paragraph{Detection}
Leveraging the \gls{ESC} telemetry values of Voltage (V), Current (I), and Rotational Speed (RPM) in addition to the \gls{IMU} data a Thrust/Power model of each motor is built.
For each control signal the predicted thrust values are used in combination to the inertial properties of the device to create an acceleration prediction vector $\mathbf{a}_{prediction}$. The actual acceleration,  $\mathbf{a}_{actual}$, is recorded using the \gls{IMU}.
\begin{equation}\label{eq:residual}
    \mathbf{r}(t) = \mathbf{a}_{prediction}(t) - \mathbf{a}_{actual}(t)
\end{equation}
The residual $\mathbf{r}(t)$ is shown in Equation \ref{eq:residual} and if the autocorrelation of $\mathbf{r}(t)$ over a sustained period it shows that there is a rotor fault or a change in the dynamics of the device. 
\paragraph{Quantification}


\paragraph{Isolation}
This means that the system is vulnerable to multiple actuator faults as all others are assumed healthy. There are methods that can address this at the cost of extra complexity \cite{ZHANG2008}. Given that the probability of simultaneous partial faults is low these are not necessary for this application, however, common mode failures could be possible for example if in sand storm you would expect multiple faults.


\end{comment}
