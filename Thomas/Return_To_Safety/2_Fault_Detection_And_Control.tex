\subsection{Fault Detection, Quantification and Control}\label{sub_section:tgt_fault_detection}

\subsubsection{State of Charge}\label{sub_sub_section:tgt_SOC}
\paragraph{Component Ageing}
Cell ageing and actuator ageing can cause inaccurate state of charge predictions. Therefore, the planned path may exceed the limit of the device leading to a state of charge fault and the necessity of \gls{RTH} if the state of charge is within 10\% of the predicted required state of charge required for \gls{RTH}. However, the state-vector telemetry recorded by the flight controller will be used to regularly inspect the cell performance and replace the cells or actuators if needed before failure.
\paragraph{Cell Failure}
\gls{LiON} batteries can have a thermal runaway. This is seen with a massive spike in temperature from the \gls{BMS} telemetry data and should lead to immediate landing and battery shutdown as it is a chain reaction effect that can cause excessive thermal damage\cite{LiONRunaway}.

\subsubsection{Actuator Fault}\label{sub_sub_section:tgt_actuator_fault}
\paragraph{Causes and Modelling}
A loss of actuator effectiveness can be either from motor damage or propeller damage. The possible causes of motor damage specific to the operating environments include: wire damage, cooling blockages and dust or sand getting into the components. For propeller damage this could be from impacts, thermal effects or fatigue cracks. In either case the actuator can be modelled as a reduced thrust/watt effect.
\paragraph{Control Strategy}
The simplest solution is to tune the output signals from the controller until the thrust of the actuator matches the expected value. This allows the control loop to be unchanged, abstracting from the physical effects. However, the actuator will saturate at a lower thrust value meaning the control loop will not perform as expected and can lead to failure in typically non-failure states. By increasing the output on the actuator it will likely cause the current defect to degenerate as the loads increase on the damaged actuator. Therefore, a gain scheduling approach is used with a selection of less aggressive controller gains to match different tuning magnitudes. As tuning magnitude increases, the controller should be less aggressive so that the thrust demands reduce. However, this means that the controller cannot tolerate disturbances of the same magnitude. Approaches such as Model Reference Adaptive Control can provide more optimal solutions however, they come at the cost of increased computing complexity and less predictable behaviour.