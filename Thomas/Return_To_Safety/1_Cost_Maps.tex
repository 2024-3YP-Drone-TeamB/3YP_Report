\subsection{Cost Maps}
\textbf{NEED FIGURES}
\subsubsection{Source data}
When considering \gls{RTS} it requires prior knowledge of the environment to be executed successfully. Therefore, this must be loaded into the drone using ground operators. The easiest and most efficient way to model this is using bird's eye images of the operating areas and marking the obstructions, suspected mined regions, mildly dangerous regions, and safe landing regions, with corresponding \gls{GNSS} co-ordinates.
\paragraph{Satellite Images}
Where satellite images are available and accurate they are the easiest and most efficient option. However, given the nature of environments effected by war, there may have been significant changes to the local environment since the last satellite image. Furthermore, seasonal changes can make satellite images less effective as trees may be less visible from winter images due to a lack of leaves.
\paragraph{Surveillance Images}
Standard consumer level drones can provide images with tagged \gls{GNSS} co ordinates that can be used, however, this requires extra hardware and time to execute. Therefore, this should be avoided where satellite images are sufficient.

\subsubsection{Graphical User Interface}
\paragraph{Platform} 
The \gls{GUI} was considered from the start to maximise usability by untrained ground operators. I built the \gls{GUI} as a simple python script that takes an image path and uses hardware-independent click locations and keystrokes to operate. It then gives he output as a .txt file with a standard format. This means that the process can be run on any local hardware available.

\paragraph{User interactions}
It is a simple program where the user can click on a region to set it a colour representing the classification. A key development was using semi-transparent colours so the base image can be seen through the shade of classification colour allows for a more natural filling experience. Furthermore, some utility functions including multi-square filling, clicking on already classified regions to deselect and auto-filling regions were added to make the process easier and faster.

\subsubsection{Tessellated surfacing}
\paragraph{Why hexagons} While squares are the easier and more natural decision, they are worse than hexagons for this application where the path planning is between central nodes. This is because, hexagons have equal distances in all directions to the centre of the next hexagon \textbf{add a figure}; while travelling from node to node in squares is >40\% further on diagonals than on the primary axes. This is difficult to interpret as the ground operator when converting to flow maps in the path planning stage. Furthermore, travelling node to node on the diagonals of squares goes through a point of four intersection where the classification is undefined \textbf{add figure}. Therefore if the user added two obstacles diagonally connected it is unknown if you could travel diagonally between them, this ambiguity creates issues that you do not face when using hexagons as while at the intersections the classifications are still undefined, when travelling node to node you never cross an intersection.

\paragraph{Mapping hexagons to \gls{GNSS}}
Talk about row, row parity, mapping to Cartesians, include a nice annotated figure (just use existing one and reference it), using reference GNSS and scale of image