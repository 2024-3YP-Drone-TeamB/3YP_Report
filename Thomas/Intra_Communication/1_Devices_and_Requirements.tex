\subsection{Design Considerations}
\subsubsection{Modularity}
\paragraph{Relevance}
Modularity improves the longevity and usability of the \gls{UAV}. This is because it allows for components to be easily replaced with either a working unit of the same kind in repairs, an upgraded unit or a unit tuned for specific mission requirements, regulations or operating conditions. Mechanical modularity is the most obvious path as the components need to physically connect easily however, the communication architecture also needs to support this so that devices can easily communicate even if they are changed completely.
\paragraph{Modular Communication Protocols}
Modular communication architectures are based around nodes. Some protocols such as \gls{SPI} operate as Master-Slaves whereas others such as Cyphal or \gls{UART} are more flexible, however, the concept remains the same. The networks allow for different devices to communicate without specific dependencies on each other. Therefore, if I was communicating to an \gls{IMU} via \gls{SPI} it would not make a difference if I then changed the \gls{IMUI} with the exception of a possible software update. Another key property of modular networks is physical, some networks lend themselves better to modular design, for example for a \gls{CAN} Bus you only need to add a stub to add a node making the physical process very easy.

\subsubsection{Telemetry}
\paragraph{Battery Telemetry}
\paragrpah{\gls{ESC} Telemetry}

\subsection{Devices}
\subsubsection{Safety Critical Devices}
\subsubsection{Required Connections}
\subsubsection{Layout}

