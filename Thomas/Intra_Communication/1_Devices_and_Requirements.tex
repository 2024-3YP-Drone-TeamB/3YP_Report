\subsection{Design Considerations}

\subsubsection{Modularity}
\paragraph{Relevance}
Mechanical modularity is the most obvious path as the components need to physically connect easily however, the communication architecture also needs to support this so that devices can easily communicate even if they are changed completely.
\paragraph{Modular Communication Protocols}
Modular communication architectures are based around nodes. Some protocols such as \gls{SPI} operate as Master-Slaves whereas others such as Cyphal or \gls{UART} are more flexible. These networks allow for different devices to communicate without specific dependencies on each other. Therefore, integrating new sensors or modules does not require a new communication system. Another key property of modular networks is physical, some networks lend themselves better to modular design, for example for a \gls{CAN} Bus you only need to add a stub to add a node making the physical process very easy.

\subsubsection{Telemetry}
\paragraph{Battery Telemetry}
The two key failure modes for the battery are state of charge and thermal runaway. Both can cause complete failure and therefore both should be monitored during flight. In the drone clearance operating environment these issues are made more extreme due expected high temperatures in heavily mined countries putting the batteries closer to thermal runaway and reduce cell efficiency \cite{REF}.
\paragraph{\gls{ESC} Telemetry}
\gls{IMU} data can be used in conjunction with command signals to approximate actuator faults, however, as it is measuring the effect is is limited in precision \cite{REF}. Therefore, providing \gls{ESC} telemetry for voltage, current and \gls{RPM} is incredibly important information when classifying motor and propeller faults accurately.   

\subsubsection{Redundancy}
\paragraph{Relevance}
Reducing communication failures requires redundant components as a single failure should not cause the communication system to break. This means that where possible, the system should be able to tolerate the failure of a single node.
\paragraph{Distributed Networks}
If a safety critical device itself cannot support redundancy it should be able to communicate on at least two different lines. This allows for a single failure to not cause common mode failure.

\subsubsection{Safety Critical Devices}
\paragraph{Relevance}
Safety critical devices ensure that the drone does not sustain damage or isolate itself. This includes components necessary for flight: flight controller, battery, \gls{ESC}s, and, the motors. It also includes the sensors required for \gls{RTS}: location sensing,  collision detection and altimetry.
\paragraph{Risk mitigation through node distribution}
Distributing the devices on different communication lines reduces the risk of failure as it reduces the risk of common mode failure. For example, if a \gls{GNSS} module were to have an improper fixture and move in a way that it severs all connections; if both communication lines are in it, both are severed. However, if it is on only one communication line then the other communication line is undamaged.
