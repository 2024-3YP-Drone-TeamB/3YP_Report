\newpage
\fancyhead[C]{Thomas Turner}
\section{Custom Hardware} \label{section:Custom Hardware}

\subsection{Introduction and Philosophy}
Custom hardware is a key novelty to our design and helps us reduce our cost and increase our performance for the specific task at hand. This is because, our custom hardware affordably mitigates the specific risks of operation over the minefield. The key risk not faced by similar commercial projects is the fact we cannot safely carry out emergency landing procedures when a fault is detected, or in fact crash when a fault is incurred due to the difficulty of retrieval. Not only does it possibly damage the imaging sensors onboard but it comes at the significant opportunity cost of not clearing the mines if the drone is lost. 
\paragraph{When to use custom components}
\gls{COTS} options should always be considered before any custom part as if there is something that already meets the design requirements it is often the better choice. This is because you do not need to invest the design and development time into making a new device and the current manufacturer benefits from greater economies of scale. However, often when choosing \gls{COTS} options they can cause compromises to cost and specification that make custom hardware appropriate in specialist applications such as this project.
\paragraph{Objectives}
All hardware should be easy and quick to debug even by unskilled ground operators, it should support redundancy in both sensors and in communications. They should also line up with our sustainability objectives by ensuring all components are \gls{RoHS} compliant and our modularity objectives by having clear and easy interfaces.


\subsection{Flight Controller}
\subsubsection{\gls{COTS} options}
\subsubsection{Component Selection}
\paragraph{\gls{IMU}}
\paragraph{\gls{MCU}}
\paragraph{\gls{CAN} Transceiver}
\subsubsection{Design Considerations}
\paragraph{Trace Lengths}
\paragraph{Touch Protection}
\paragraph{Component Placement}
\paragraph{Headers and interfaces}
\paragraph{Trace Widths}
\paragraph{Layers}
\subsubsection{Schematic Design}
\subsubsection{Component Placement}


\subsection{GNSS Module}
\subsubsection{\gls{COTS} options}
\subsubsection{Component Selection}
\subsubsection{Design Considerations}
\subsubsection{Schematic Design}
\subsubsection{Component Placement}