\subsection{Design Considerations}

\subsubsection{Trace Lengths}
\paragraph{Signal Integrity and Timing}
All traces have propagation delay that increases linearly with length. This means that the longer the trace, the longer the delay. Furthermore, the longer the length, the greater effect impurities and crosstalk have creating a nosier signal. Therefore, traces should be kept as short as possible.
\paragraph{Length Matching}
In synchronous communication methods the clock signal must be in sync with the data transmission. This means that the lines should be as similar in length as possible so that they arrive at the same delay. The faster the communication rate the stricter this must become as the clock signal frequency is increased. Therefore, signal speeds should be as low as possible to increase robustness.
\paragraph{Effect of vias}
Vias allow for the transference of signals between planes which is often necessary for routing. However, they introduce extra impedance, noise and delay \cite{source} and therefore should be avoided. Furthermore, for synchronous high speed signals they should be included in trace length considerations as the line with fewer vias should be tuned to have a longer length to achieve the same delay. For this application, as none of the signals are greater than \textbf{highest speed} tuning exact timings is not necessary.

\subsubsection{Touch Protection}
\paragraph{\gls{ESD}}
\gls{ESD} occurs when there is a significant static charge built up in a ground operator or contacting surface that is then shorted in contact with the board. As electro-static voltages can be \cite{REF} this can cause significant damage to devices.
\paragraph{Protection Devices}
In order to protect against \gls{ESD} the circuit must provide a low resistance pathway to the ground. This can be using \gls{TVS} diodes that above certain voltages act like a wire that diverts the current flow. They can also be mitigated using voltage dependent resistors that increase resistance with voltage and absorb the errors. Lastly, by using low resistance resistors in series it decreases the current spike. In my designs I make use of both \gls{TVS} diodes and series resistors due to their simplicity and effectiveness.
\paragraph{Component Placement}
\gls{ESD} is largely from interactions with people, therefore, efforts to mitigate the effects should be as close to these expected points as possible. This is because the current loop to the ground should be as small as possible in order to minimise the damage to components. Therefore, in my designs \gls{TVS} diodes were placed near all interfaces and removable memory devices as these are the most likely points of contact.

\subsubsection{Component Placement}
\paragraph{Utility Regions}
Components with similar functionalities should be kept separate. For example, all digital signals should be kept away from analogue signals as the high frequency digital signals can induce noise in the analogue signals. In my design there are only digital signals and therefore the key areas are between the power management systems, sensors and communications that are all on separate board areas.
\paragraph{Thermal Considerations}
Heat dissipating elements, typically diodes and resistors, can cause damage to electronic components if not sufficiently managed. Therefore, heat sinks or controlled airflows are required if there are high power draws. This consideration is why on my devices Buck Converters of above 90\% \cite{REF} are used instead of the less efficient \gls{LDO} which would have an efficiency of 66\% when converting from 5V to 3.3V. This, in addition to the low power draws of all other components, means that no explicit thermal management is needed.

\subsubsection{Trace Widths and Spacing}
\paragraph{Impedance Control and Thermal Management}
The impedance of a trace is proportional to its cross sectional area as it acts like a wire. Since the trace thickness standard as one ounce per square foot it is the cheapest to manufacture with and is used. Therefore, to control the impedance of a trace the width must be calibrated. \textbf{a nice table showing the widths and current capacities}
\paragraph{Crosstalk}
When traces are too close too each other they can induce signals in each other. This means that traces should be spaced appropriately. However, since all the signals are digital or physically isolated on the board this is not a major concern but I did ensure that all traces are at least 1.5 trace widths separated as a form of best practice. 

\subsubsection{Layers}
\paragraph{Two Layers}
The simplest and most cost effective option in \gls{PCB} design is two copper layers separated by a dielectric. By using a copper filled regions you can create a ground plane and a power plane that effectively distribute charge and maintain voltage integrity. The simplicity of the \gls{GNSS} module means that this is the best and most cost effective option. This comes at the cost of more difficult component placement and a slightly less stable power and ground plane. However, this instability can be mitigated using distributed ceramic capacitors between the planes that filter high frequency noise in voltage levels.
\paragraph{Four Layers}
When circuits become more complex, a purpose ground and power plane in between the top and bottom plane can allow for greater packing density of components. They also make the planes more stable as all the path impedances are very low. This means that for complex designs with sensors, like the flight controller module, four layer \gls{PCB}s are the best option.
\paragraph{Six or more Layers}
If a circuit contains many high frequency analogue signals it can be beneficial to have separate ground and power planes and indeed extra signal planes to reduce the effects of crosstalk and to maintain stable voltages. However, specific to this use case, this added complexity is not necessary and would act as an extra cost for very limited benefits.


