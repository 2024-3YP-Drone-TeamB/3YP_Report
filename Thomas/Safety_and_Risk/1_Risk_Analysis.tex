\subsection{Risk Analysis}
\subsubsection{Ground Operator Owned Risks}
% maybe I want to include injury and more safety related risks?
\begin{table}[h]
\begin{tabular}{|>{\raggedright\arraybackslash}p{5cm}|c|c|c|>{\raggedright\arraybackslash}p{5cm}|}
\hline
\textbf{Risk Description} & \textbf{Likelihood} & \textbf{Impact} & \textbf{Severity} & \textbf{Mitigation Actions} \\ \hline
\textbf{Sensor Malfunction or Degradation}: Imaging sensors may degrade affecting detection capability. & \HighRisk      & \HighRisk & \HighRisk & Regular maintenance and calibration. \\ \hline
\textbf{False Negatives}: Landmines not detected if obscured by noise or adverse conditions. & \MediumRisk & \HighRisk & \HighRisk & Use a multi-layered sensor approach. \\ \hline
\textbf{False Positives}: Debris misclassified as landmines, leading to wasted clearance time.& \HighRisk & \LowRisk    & \MediumRisk  & Apply data fusion and secondary confirmation methods. \\ \hline
\textbf{Weather}: Extreme rain, wind, or hail can cause damage and loss of control. & \HighRisk & \HighRisk & \HighRisk & Monitor pre-mission weather forecasts; enable a \gls{RTH} trigger via LoRa. \\ \hline
\textbf{Regulatory and Privacy Concerns}: Issues arising from aviation disruption and privacy. & \MediumRisk & \MediumRisk & \MediumRisk  & Collaborate with local authorities and host public meetings to ensure regulatory compliance and community support. \\ \hline
\textbf{Device Field Failure}: Poor operator use or a module failure leaves the device inoperable. & \VeryHighRisk & \LowRisk & \HighRisk & Use a flight controller LED array to isolate issues and design modular components for quick replacement. \\ \hline
\textbf{Hardware ageing}: Sensors and actuators become inaccurate or inefficient. & \VeryHighRisk & \MediumRisk & \HighRisk & Use pre-loaded test command sequences to regularly test and recalibrate sensors and actuators.\\ \hline
\end{tabular}
\caption{Risk Register: Ground Operators}
\label{tab:risk_register_ground_operator}
\end{table}

 
\paragraph{Field Testing}
Ground operators can load in tests that have known expected results. These tests are formed of control signals and then sensor measurements. If everything is working then the results match the expected results and the device is operable. If there is an actuator fault the sensors will all record similar results different to the expected results and if there is a sensor fault the sensors will all record the expected results except for the faulty sensors. Furthermore, all communication lines are tested to check they are operable even if they are redundant. This isolates any module faults so that they can be replaced and sent for repair.
\paragraph{Role of Design}
The design of the device should support ground operators at all times. Therefore, clear and easy documentation will be provided and all interfaces are as simple and easy to use as possible. However, in order to continue to support progress feedback must be gathered to improve the design in a testing phase and throughout operation.

\subsubsection{Designer Owned Risks}
\begin{table}
\begin{tabular}{|>{\raggedright\arraybackslash}p{5cm}|c|c|c|>{\raggedright\arraybackslash}p{5cm}|}
\hline
\textbf{Risk Description} & \textbf{Likelihood} & \textbf{Impact} & \textbf{Severity} & \textbf{Mitigation Actions} \\ \hline
\textbf{Communication Bus Failure}: Bus is severed or not securely connected. & \MediumRisk & \HighRisk & \HighRisk & Utilize redundant communication lines and pre-flight test signals. \\ \hline
\textbf{Hardware Geolocation Failure}: Failure of the GNSS module. & \MediumRisk & \HighRisk & \HighRisk & Install a redundant GNSS module. \\ \hline
\textbf{State of Charge Failure}: Device out of power mid-flight, leading to emergency landing or crash. & \MediumRisk & \HighRisk & \HighRisk & Implement a robust \gls{BMS} with automatic \gls{RTH} when below threshold. \\ \hline
\textbf{Cybersecurity Threats}: Blocked GNSS or spoofed LoRa communication. & \LowRisk & \HighRisk & \MediumRisk & Limit LoRa functions to triggering \gls{RTH}; a GNSS module can perform dead reckoning. \\ \hline
\textbf{Partial Thrust Loss}: Damage to a propeller or motor causes reduced effectiveness in a propeller-motor set. & \HighRisk & \MediumRisk & \HighRisk & Deploy adaptive control techniques with \gls{RTS} measures. \\ \hline
\textbf{Flight Controller Failure}: no control signal for the device. & \LowRisk & \HighRisk & \MediumRisk  & Independent control capability with GNSS modules. \\ \hline
\end{tabular}
\caption{Risk Register: Designers}
\label{tab:risk_register_designer}
\end{table} 
\paragraph{Designed Redundancy}
Reducing the risk of failure through redundancy is a key mitigation to the primary high severity risks as shown in \ref{tab:risk_register_designer}. This, in combination with regular field testing, means that the risk of failure is a low as reasonably possible as all modules and redundant modules are tested regularly so it would require a common mode failure, a catastrophic failure or a very unlikely dual failure.
\paragraph{Analysis and updates}
During testing and deployment if there are failures, they need to be analysed and mitigated against. This requires a designed level of support for analysis, improvement and replacement. For analysis, there is a removable memory device for all telemetry and error messages in addition to easy debugging methods. For improvements, all designed hardware have the specifications to increase their level of support if needed. Finally for replacement, all modules can be replaced easily with different specification versions. This design level flexibility is vital for the longevity of the design.
