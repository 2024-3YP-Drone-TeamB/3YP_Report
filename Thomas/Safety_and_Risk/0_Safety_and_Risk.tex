\section{Safety and Risk}\label{Safety and Risk}
\subsection{Introduction}\label{sub_section:tgt_safety_intro}
Safety and risk are key considerations for this project. They are considered for ground operators, civilians who are negatively impacted by landmines, and the business. High-speed impacts are the most likely to affect all three, as impacts will damage the sensors and body as well as anything that it collides with. However, concerning the business and civilians, downtime is also a significant factor, as the longer the drone is inactive, the higher the cost to demine an area and the more likely for injury, death, and obstruction to civilians. Therefore, we aim to minimise the risk of high-speed collisions whilst ensuring that the drone is not stranded in unreachable locations, and repairs are quick and easy.
\subsection{Risk Analysis}\label{sub_section:tgt_risk}
\subsubsection{Ground Operator Owned Risks}\label{sub_sub_section:tgt_ground_operator_risk} \begin{table}
  \centering
  \begin{tabular}{|c|c|c|c|c|}
    \hline
    \textbf{} & \textbf{Low Impact} & \textbf{Medium Impact} & \textbf{High Impact}& \textbf{Very High Impact} \\
    \hline
    \textbf{Very High Risk}  & \HighRisk & \HighRisk   & \VeryHighRisk & \VeryHighRisk \\
    \hline
    \textbf{High Risk}  & \MediumRisk & \HighRisk   & \HighRisk & \VeryHighRisk\\
    \hline
    \textbf{Medium Risk}  & \LowRisk & \MediumRisk   & \HighRisk & \HighRisk\\
    \hline
    \textbf{Low Risk}  & \LowRisk & \LowRisk   & \MediumRisk & \HighRisk\\
    \hline
  \end{tabular}
    \caption{Risk Assessment Matrix}
  \label{tab:risk-matrix}
\end{table}

% maybe I want to include injury and more safety related risks?
\begin{table}[h]
\begin{tabular}{|>{\raggedright\arraybackslash}p{5cm}|c|c|c|>{\raggedright\arraybackslash}p{5cm}|}
\hline
\textbf{Risk Description} & \textbf{Likelihood} & \textbf{Impact} & \textbf{Severity} & \textbf{Mitigation Actions} \\ \hline
\textbf{Sensor Malfunction or Degradation}: Imaging sensors may degrade affecting detection capability. & \HighRisk      & \HighRisk & \HighRisk & Regular maintenance and calibration. \\ \hline
\textbf{False Negatives}: Landmines not detected if obscured by noise or adverse conditions. & \MediumRisk & \HighRisk & \HighRisk & Use a multi-layered sensor approach. \\ \hline
\textbf{False Positives}: Debris misclassified as landmines, leading to wasted clearance time.& \HighRisk & \LowRisk    & \MediumRisk  & Apply data fusion and secondary confirmation methods. \\ \hline
\textbf{Weather}: Extreme rain, wind, or hail can cause damage and loss of control. & \HighRisk & \HighRisk & \HighRisk & Monitor pre-mission weather forecasts; enable a \gls{RTH} trigger via LoRa. \\ \hline
\textbf{Regulatory and Privacy Concerns}: Issues arising from aviation disruption and privacy. & \MediumRisk & \MediumRisk & \MediumRisk  & Collaborate with local authorities and host public meetings to ensure regulatory compliance and community support. \\ \hline
\textbf{Device Field Failure}: Poor operator use or a module failure leaves the device inoperable. & \VeryHighRisk & \LowRisk & \HighRisk & Use a flight controller LED array to isolate issues and design modular components for quick replacement. \\ \hline
\textbf{Hardware ageing}: Sensors and actuators become inaccurate or inefficient. & \VeryHighRisk & \MediumRisk & \HighRisk & Use pre-loaded test command sequences to regularly test and recalibrate sensors and actuators.\\ \hline
\end{tabular}
\caption{Risk Register: Ground Operators}
\label{tab:risk_register_ground_operator}
\end{table}


\paragraph{Boot Testing}
Before the start of any mission, the flight controller gives standard control signals directly after take-off. If the sensor recordings match the expected results for the control signals, the mission goes ahead. If there is an actuator fault, the sensors will all record similar results different from the expected results, and if there is a sensor fault, the sensors will all record the expected results except for the faulty sensor. Furthermore, all communication lines are tested to check they are operable, even if they are redundant. This isolates any module faults so that they can be replaced and sent for repair.
\paragraph{Role of Design}
The design of the device should support ground operators at all times. Therefore, clear and easy documentation will be provided, and all interfaces are as simple and easy to use as possible. However, in order to continue to support progress, feedback must be gathered to improve the design in a testing phase and throughout operation.

\subsubsection{Designer Owned Risks}\label{sub_sub_section:tgt_design_risk}
\begin{table}[h]
\begin{tabular}{|>{\raggedright\arraybackslash}p{5cm}|c|c|c|>{\raggedright\arraybackslash}p{5cm}|}
\hline
\textbf{Risk Description} & \textbf{Likelihood} & \textbf{Impact} & \textbf{Severity} & \textbf{Mitigation Actions} \\ \hline
\textbf{Communication Bus Failure}: Bus is severed or not securely connected. & \MediumRisk & \HighRisk & \HighRisk & Utilize redundant communication lines and pre-flight test signals. \\ \hline
\textbf{Hardware Geolocation Failure}: Failure of the GNSS module. & \MediumRisk & \HighRisk & \HighRisk & Install a redundant GNSS module. \\ \hline
\textbf{State of Charge Failure}: Device out of power mid-flight, leading to emergency landing or crash. & \MediumRisk & \HighRisk & \HighRisk & Implement a robust \gls{BMS} with automatic \gls{RTH} when below threshold. \\ \hline
\textbf{Cybersecurity Threats}: Blocked GNSS or spoofed LoRa communication. & \LowRisk & \HighRisk & \MediumRisk & Limit LoRa functions to triggering \gls{RTH}; a GNSS module can perform dead reckoning. \\ \hline
\textbf{Partial Thrust Loss}: Damage to a propeller or motor causes reduced effectiveness in a propeller-motor set. & \HighRisk & \MediumRisk & \HighRisk & Deploy adaptive control techniques with \gls{RTS} measures. \\ \hline
\textbf{Flight Controller Failure}: no control signal for the device. & \LowRisk & \HighRisk & \MediumRisk  & Independent control capability with GNSS modules. \\ \hline
\end{tabular}
\caption{Risk Register: Designers}
\label{tab:risk_register_designer}
\end{table} 
\paragraph{Designed Redundancy}
Reducing the risk of failure through redundancy is a key mitigation to the primary high severity risks as shown in Table \ref{tab:risk_register_designer}. This, in combination with regular field testing, means that the risk of failure is a low as reasonably possible, as all modules and redundant modules are tested regularly, so it would require a common mode failure, a catastrophic failure, or a dual failure.
\paragraph{Analysis and updates}
During testing and deployment, if there are failures, they need to be analysed and addressed. This requires a designed level of support for analysis, improvement, and replacement. For analysis, there is a removable memory device for all telemetry and error messages, in addition to easy debugging methods. For improvements, all designed hardware has the specifications to add functionality by ensuring it exceeds the current requirements. Finally, for replacement, all modules can be replaced easily with different specification versions. This design level flexibility is vital for the longevity of the design. Furthermore, to ensure that the device is as affordable as possible, the risk matrices are re-evaluated regularly with the deployment data to isolate areas that are overly mitigated. For example, if it is found that the \gls{GNSS} module has a negligible chance of failure, the redundant \gls{GNSS} modules can be removed from the device to reduce unit cost, allowing for the further reduction in demining costs.
\paragraph{Intellectual Property}
 To maintain the standing of the company and ensure quality in the company's products, there will be a trademark in place for the company name and logo. However, for ethical reasons, as explored in Section \ref{sustainability}, all designs and underlying technology will be open-source to allow for the most innovation in this critical field.
\subsection{Safety}\label{sub_section:tgt_safety}
\subsubsection{Manufacture}\label{sub_sub_section:tgt_safety_manufacture}
\paragraph{Incidence Reporting}
It is important throughout manufacturing and assembly that all incidents, near misses, and mistakes are reported to reduce future incidents. Therefore, there will be an anonymous incident reporting form in addition to a supportive culture. Furthermore, these incidents and feedback will be collected from ground operators to find areas where the design can support safety.
\paragraph{\gls{HAZOP}}
To ensure that all possible hazards and opportunities for improvement are considered, there will be monthly \gls{HAZOP} meetings in addition to an initial \gls{HAZOP} study. The attendees will include stakeholders, users, designers, and manufacturers to ensure every level is addressed regularly.

\subsubsection{Operation}\label{sub_sub_section:tgt_safety_operation}
\paragraph{Clearance}
All devices are equipped with obstacle detection modules, and if they detect any objects within 10 meters, they will not take off. This is to ensure that all people and possible obstructions are cleared to prevent collisions and harm. Furthermore, before landing, the drone will hover for 3 seconds at 3 meters above ground level to ensure time for operators to clear the landing zone.
\paragraph{Retrieval}
Retrieval from mined areas is incredibly dangerous; therefore, the custom \gls{RTS} system was developed as discussed in Section \ref{Return to Safety}. This reduces the risk that the device is left stranded. The only way the drone can be retrieved is by clearing a path to it, which introduces significant delay; however, using the multiagent system in combination with traditional methods, it is possible. Lastly, to ensure that the device does not crash near ground operators while executing \gls{RTS}, the base station is always listed as an obstacle that cannot be visited.


\subsection{Conclusion}
Where possible, high severity risks are mitigated with design-level interventions and the device uses design to support ground operator risk reduction. Both the operator safety and the manufacturer safety are considered at the design stage and throughout to ensure that all incidents are investigated and hazards are addressed. However, all risk and safety analysis will benefit from test and deployment data for further improvements.