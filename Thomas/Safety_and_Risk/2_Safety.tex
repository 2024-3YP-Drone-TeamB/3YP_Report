\subsection{Safety}\label{sub_section:tgt_safety}
\subsubsection{Manufacture}\label{sub_sub_section:tgt_safety_manufacture}
\paragraph{Incidence Reporting}
It is important throughout manufacture and assembly that all incidents, near misses and mistakes are reported in order to reduce future incidents. Therefore, there will be an anonymous incident reporting form in addition to a supportive culture. Furthermore, this incidents and feedback will be collected from ground operators to find areas where the design can support safety.
\paragraph{\gls{HAZOP}}
To ensure that all possible hazards and opportunities for improvement are considered there will be monthly \gls{HAZOP} meetings in addition to an initial \gls{HAZOP} study. The attendees will include stakeholders, users, designers and manufacturers to ensure every level is addressed regularly.

\subsubsection{Operation}\label{sub_sub_section:tgt_safety_operation}
\paragraph{Clearance}
All devices are equipped with obstacle detection modules and if they detect any objects within 5 meters, they will not take off. This is to ensure that all people and possible obstructions are cleared to prevent collisions and harm. Furthermore, before landing the drone will hover for 3 seconds at 3 meters above ground level to ensure time for operators to clear the landing zone.
\paragraph{Retrieval}
Retrieval from mined areas is incredibly dangerous therefore the custom \gls{RTS} system was developed. This reduces the risk that the device is left stranded. The only way the drone can be retrieved is by clearing a path to it which introduces significant delay however, using the multiagent system in combination to traditional methods it is possible. Lastly, to ensure that the device does not crash near ground operators while executing \gls{RTS} the base station is always listed as an obstacle that cannot be visited.