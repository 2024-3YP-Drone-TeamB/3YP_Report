\newpage
\fancyhead[C]{Huirui Dai}
\setlength{\parindent}{0pt}

\section{Sustainability and Ethics} \label{sustainability_ethics}


\subsubsection{Sustainability}\label{sustainability}

The environmental sustainability of the proposed \gls{UAV}-based landmine detection system can be evaluated through three primary aspects: environmental impact, hardware modularity, and energy management. Taken together, these design choices contribute to a system architecture that not only supports operational effectiveness but also aligns with core principles of environmental sustainability.

\paragraph{Environmental Impact}

\gls{UAV} operations inevitably introduce certain environmental disturbances, like noise pollution that may disrupt local wildlife. Additionally, large-scale drone deployment increases energy consumption. However, these drawbacks are outweighed by the substantial environmental benefits gained through improved landmine detection efficiency. By accelerating the clearance process, our drone system reduces the duration that landmines remain embedded in the soil, thereby mitigating their serious long-term environmental damage, like degradation caused by chemical contamination, loss of biodiversity, or reduction in productivity due to restricted land for agriculture~\cite{eniang2007impacts}.

\paragraph{Hardware Modularity}

As described in Section~\ref{sub_sub_section:tgt_modularity}, the modular design of our communication subsystem allows for hardware changes without requiring reconfiguration of the core communication protocol. This facilitates the integration of new sensors or peripherals, improving adaptability over time. Also, the \gls{RTH} and \gls{RTS} outlined in Section~\ref{Return to Safety} ensures that drones are retrievable after mission failures. So functioning modules like GNSS, ESCs, or battery packs can be reused in new builds. The crash logs also provide diagnostic data, allowing engineers to identify failure modes and implement targeted upgrades, thereby enhancing future system robustness. This circular approach minimizes hardware waste and promotes a sustainable development cycle.

\paragraph{Energy Management}

In our project, a custom-designed \gls{BMS}, detailed in Section~\ref{sec:bms}, was developed to extend battery longevity and reduce the frequency of cell replacement. We selected high-reliability Li-io cells from Overlander, which offer improved resistance to thermal runaway and a higher cycle life compared to standard cells. Additionally, our power management strategy includes an integrated thermal regulation system that maintains batteries within an optimal operating temperature range, which enhances the overall lifetime and recyclability of the power supply.


\subsubsection{Ethics}\label{ethics}

The primary ethical motivation behind this project is humanitarian: to protect civilians—especially vulnerable populations such as children—from the persistent threat of \gls{APL}s. To support this goal, our \gls{UAV} system is designed with multiple safety mechanisms. Each drone is equipped with obstacle avoidance sensors to prevent collisions with people or objects, and a \gls{RTS} protocol ensures that drones do not crash in dangerous mine fields~\ref{sub_sub_section:tgt_safety_operation}. Furthermore, as discussed in Section~\ref{compvis_intro}, our algorithm emphasizes recall over precision. This priority of detecting all possible landmines, even at the risk of false positives, ensures that no hazardous areas are overlooked, thereby protecting the lives of demining personnel and civilians.

To prevent misuse and protect sensitive information, our drone system is explicitly designed for humanitarian demining and does not support any form of weaponization. The platform is marketed exclusively to authorized demining agencies, reducing the likelihood of military usage. Additionally, data privacy is taken seriously. Our system incorporates encrypted communication and storage protocols, and since our services are typically contracted by governmental or internationally recognized demining organizations, the risk of sensitive geographic data leaking to unauthorized third parties is minimal.

Finally, the project complies with relevant legal and ethical standards. As detailed in Table~\ref{tab:risk_register_ground_operator} on risk assessment, we work closely with local authorities and communities to ensure compliance with aviation laws, safety protocols, and public privacy expectations. All custom hardware components used in the drone platform are \gls{RoHS} compliant~\ref{sub_section:tgt_custom_hardware_intro}. By adhering to these standards, the project maintains responsible engineering practices while supporting inclusive and lawful deployment.