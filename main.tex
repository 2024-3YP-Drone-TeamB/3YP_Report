\documentclass{report_template_oxford}

\newacronym{3yp}{3YP}{Third-Year Project}

\title{3YP report template}
\author{Daniele De Martini}
\date{March 2025}

\begin{document}
\maketitle

% Abstract
% - Concise summary of report
% - Naming of essential outcomes
% - As short as possible while providing key points
% - Less than 1 page
\pagenumbering{roman}
\newpage
\addcontentsline{toc}{section}{Abstract}
\fancyhead[C]{Student author of section}
\section*{Abstract}

\lipsum[1]

\newpage
\fancyhead[C]{}
\tableofcontents

\newpage
\addcontentsline{toc}{section}{List of Figures}
\listoffigures

\newpage
\addcontentsline{toc}{section}{List of Tables}
\listoftables

\newpage
\addcontentsline{toc}{section}{Glossary}
\printglossaries

% Introduction
% - Provide background and motivation
% - Provide boundaries and goals to the project
% - Give a short overview of report structure
\newpage
\pagenumbering{arabic}
\fancyhead[C]{Student1 author of section}
\section{Introduction}
\lipsum[1-5]

\begin{figure}
    \centering
    \includegraphics[width=0.5\linewidth]{example-image-duck}
    \caption{An awesome image of a duck.}
    \label{fig:enter-label}
\end{figure}

\fancyhead[C]{Student2 author of section}
\subsection{An awesome image}

\lipsum[1-2]


\subsection{An awesome table}

\begin{table}[h]
    \centering
    \begin{tabular}{@{}lll@{}}
        \toprule
        \textbf{Fruit} & \textbf{Color} & \textbf{Average Weight (g)} \\ \midrule
        Apple          & Red            & 182                        \\
        Banana         & Yellow         & 118                        \\
        Orange         & Orange         & 131                        \\
        Grape          & Purple         & 5                          \\
        Watermelon     & Green          & 9000                       \\ \bottomrule
    \end{tabular}
    \caption{An awesome table, by GPT-4o mini \cite{achiam2023gpt}.}
    \label{tab:my_label}
\end{table}

\lipsum[1-6]

\subsection{An awesome Acronym}

You can use acronyms for your \gls{3yp} exactly as I did here.
You can define them in the header of this file and they will appear in the glossary section.




\bibliographystyle{plain}
\bibliography{biblio}

% Appendix
% Essential content that interrupts the flow of the document can be placed here, e.g. technical drawings, detailed lists of equations used, etc.
\newpage
\addcontentsline{toc}{section}{Appendix}
\section*{Appendix}

\end{document}
