\fancyhead[C]{Jihwan Shin}
\section{Introduction} \label{introduction}

\subsection{Motivation}

- 110mil mines buried world wide
- 6000 casualties per year
- 84percent civilians, over a third are children

- current demining techniques are mostly manual: metal detectors and other handheld devices
- 5mn usd to clear 1km2
- ukraine needs over 750 years to remove the mines planted in the last 3 years using such manual techniques

- automated systems are still being developed using UAV (give few example links)
- we want to establish a multiaerial drone system for landmine detection to accelerate this process

\cite{icbl2024landmine}
\cite{globsec2024ukraine}

\subsection{Mission}

- mission: expedite the detection and mapping of AP landmines through an autonomous drone system, improving safety, efficiency, and accessibility in demining operations worldwide

% key objectives
- Design and develop a coordinated fleet of autonomous drones capable of covering large terrains effectively.
- Create software that integrates multimodal sensor data and terrain analysis to predict landmine locations with high accuracy.
- Optimize the system to balance cost, operational autonomy, and detection precision, ensuring feasibility for deployment in diverse environments.

% key deliverables
- Key hardware component selections for sensor, communication, battery management, etc. (reference section-huirui, thomas, sam, jihwan)
- Computer vision proof of concept (section-rory)
- Mission planning and strategy proof of concept (section-jihwan)
- Optimal control simulation (section-sam)
- Failure consideration and mitigation (section-thomas)

... will be explored while applying these design philosophy
- Modular
- Cost-Efficient
- Easy-to-Use
- Robust
- Repairable
- Sustainable
