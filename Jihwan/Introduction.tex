\fancyhead[C]{Jihwan Shin}
\section{Introduction} \label{introduction}

The use of \gls{APL} remains a persistent threat to welfare worldwide. According to \cite{icbl2024landmine}, at least 5,757 casualties of landmines and explosive remnants of war were recorded for 2023. Civilians made up 84\,\% of the recorded casualties, out of which 37\,\% were children. 

Since Russia's invasion of Ukraine in February 2022, Ukraine has become the most mined country in the world \cite{globsec2024ukraine}. 30\,\% of Ukraine's territory needs to be inspected for mines (over 174,000\,km$^2$), and it is expected to take up to 757 years to demine the potentially contaminated territory with the available resources (given that they are actually contaminated) as of 2023. The average market rate of mine clearance is about 4,000,000\,USD/km$^2$ \cite{globsec2023ukraine}, indicating a challenge in both time and cost to combat the landmine crisis. 

The mission of this \gls{3YP} is to expedite the detection and mapping of \gls{APL} through an autonomous multi-aerial drone system, improving safety, efficiency and accessibility in demining operations worldwide. Throughout the report, we aim to follow a consistent design philosophy to achieve a high standard. Our designs are made to be \textbf{modular}, \textbf{cost-efficient}, \textbf{easy-to-use}, \textbf{robust}, \textbf{repairable} and \textbf{sustainable}.

We start by explaining the different types of landmines and detection techniques to decide on a multi-modal sensor system using thermal (FLIR Vue Pro) and \gls{GPR} (SPH Engineering Zond Aero 1000 MHz) sensors for data collection. Exact configurations and additional processing techniques (such as \gls{SAR}) of the sensors are also chosen to maximise coverage efficiency (Section \ref{sensor_hardware_data_acquisition}). 

The mission planning framework adopting the \textit{Layered Approach} is designed to optimise the trade-off between thermal and \gls{GPR} sensors. The framework first deploys thermal drones to cover the defined \gls{ROI} with a \gls{CPP} algorithm to identify the suspected landmine points with high recall. Next, \gls{GPR} drones are deployed to target those points with a \gls{TSPO} algorithm. This framework is expanded to a multi-drone system using the \textit{Divide and Cluster} method for a scalable solution (Section \ref{sec:msp}). 

The analysis of sensor data incorporates a computer vision model (YOLOv11) to identify the landmine locations. Thermal and \gls{GPR} simulations (using MATLAB and gprMax) are explored and implemented to improve the robustness of the system (Section \ref{computervision}). The results are then combined with a fusion algorithm integrating physics-based constraints (ANFIS) to outperform traditional methodologies (Section \ref{fusion}). 

The specifics on the drone hardware are designed with careful considerations of our design philosophy. The internal communication framework is built with a dual \gls{CAN} bus design supporting modularity and redundancy (Section \ref{Intra Communication}), and the external communication framework uses \gls{LORA} technology for a low-power solution (Section \ref{sec:euc}). Custom \gls{GNSS} and flight controller modules are designed to reduce the costs of the system while offering extended redundancy (Section \ref{section:Custom Hardware}). To meet the power requirements of the complex control system and high-quality sensors, we design and simulate a \gls{BMS} (on MATLAB and Simulink) for a chosen battery (Overlander 4500mAh 21700 Li-ion Cell) which validates its effectiveness (Section \ref{sec:bms}).

We select the cascade PID controller with a tuning process that accounts for the weather data to create a robust control system for the drones. Simulations are implemented (using MATLAB and Simulink) for single and multi-agent systems to predict the mission performance. Online estimation of incoming gust occurrences using machine learning techniques (SMOTE) is also considered and tested as a further extension (Section \ref{sec:simcontrol}). 

To combat the high risks associated with our landmine detection applications, a novel emergency protocol to safely retrieve the drones in case of hardware failure is introduced. It uses an online path planning framework with a pre-defined cost map to direct the compromised drone to a safe-to-retrieve region (Section \ref{Return to Safety}).

Finally, the non-technical details of the project in the scope of Engineering in Society are explored in the final sections. We consider the Safety and Risk (Section \ref{Safety and Risk}), Project Financing (Section \ref{financing}), Technology Strategy (Section \ref{sec:techstrat}), Sustainability and Ethics (Section \ref{sustainability_ethics}). Our multiagent autonomous drone system is estimated to reduce the operational costs by a factor of over 20 whilst being significantly faster. 

% We present the key findings for the areas of focus outlined in Table~\ref{tab:intro_focus}. Some of our designs are then tested in the context of Ukraine through simulations to further validate its applicability in current situations.  

% \begin{table}[h!]
%     \centering
%     \begin{tabular}{|p{0.6\linewidth}|c|}
%     \hline
%         \textbf{Area of Focus} & \textbf{Sections} \\
%     \hline\hline
%         Research and selection of components for a multi-modal (thermal and \gls{GPR}) sensor system to collect the field data. & \ref{sensor_hardware_data_acquisition} \\
%     \hline
%         Design of a mission planning framework to efficiently deploy the multi-aerial drone system. & \ref{sec:msp} \\
%     \hline
%         Implementation of computer vision and sensor fusion software utilising machine learning methods to detect the landmines. & \ref{computervision}, \ref{fusion} \\
%     \hline
%         Design and selection of communication, battery management and custom hardware systems to construct the drones. & \ref{Intra Communication}, \ref{sec:euc}, \ref{section:Custom Hardware}, \ref{sec:bms} \\
%     \hline
%         Design and simulation of the control system to ensure its robustness in various weather conditions. & \ref{sec:simcontrol} \\
%     \hline
%         Design of the emergency protocol to safely retrieve the drones in case of hardware failure. & \ref{Return to Safety} \\
%     \hline
%         Analysis of non-technical details of the project in the scope of Engineering in Society. & \ref{Safety and Risk}, \ref{financing}, \ref{sec:techstrat}, \ref{sustainability_ethics} \\
%     \hline
%     \end{tabular}
%     \caption{Areas of Focus}
%     \label{tab:intro_focus}
% \end{table}