\subsection{Computer Vision Implementation} \label{compvis_implementation}

\subsubsection{Datasets comaprison}

    \paragraph{Thermal}Dataset from Roboflow\footnote{\url{https://universe.roboflow.com/xamurani/landmines_detection}} of 1020 images of landmines. The set of images where taken from different altitudes (1-10m), and the mine depth varied between 0 and 5cm, and mines of different type and diameter were used. This is clearly quite robust/varied/diverse/heterogeneous/variable/complex.
    
    This is good because a diverse dataset exposed the network to a wider range of features. This allows the network to learn more generalisable features \textbf{WHY?}.
    
    \paragraph{Radar}

\subsubsection{Architecture Comparison}

    \paragraph{Thermal} The thermal image landmine detection algorithm will be a  convolutional neural network. CNNs are used because they excel in extracting low level patterns from data, and are thus quite robust to noise, and discrepancies between real life and training data. They are also computationally efficient for small models (REF XX), although we are not constrained by computational efficiency (SEC XX).
    
    The specific model used for training was a YOLOv11 large model, and had 631 layers, and  25311251 parameters
    
    \paragraph{Radar} The radar computer vision model was developed using a YOLOv11 Large architecture, which comprises 357 layers and contains 25\,311\,251 parameters. The model was trained on raw Bscans due to the lack of publicly available datasets for back-projected Bscans in ground-penetrating radar (GPR) readings of landmines\footnote{\url{https://universe.roboflow.com/hassan-sdqop/gpr-data-n3n5u}}. Training was performed on a T4 GPU using Google Colab. Although the back-projection algorithm transforms the raw Bscans into a coordinate system where the landmine features are more distinct, thereby potentially enhancing detection performance, the absence of a sufficiently diverse back-projected dataset meant that raw Bscans were used instead. After consultation with Huirui, it was agreed that models trained and tested on raw Bscans would provide a conservative performance estimate, effectively serving as a lower bound relative to models trained on optimally processed back-projected images. Such back-projected data would likely require extensive real-world experiments to source.

\subsubsection{Training}

    \paragraph{Thermal} Trained for 100 epochs.
    Trained on a Tesla T4 GPU rented from google colab (REF XX)
    Trained in 1.153 hours

    The average preprocessing, inference and postprocessing times was 1.3ms, 14.9ms and 3.6ms respectively.
    
    \paragraph{RADAR} Training was completed over 100 epochs, taking approximately 0.156 hours in total. 

\subsubsection{Results}

    \paragraph{Thermal}     Precision/Recall Curve: 
    
    \paragraph{RADAR} Preliminary results indicate that the model is capable of recognising landmine signatures in noisy raw Bscan images, though its performance is expected to improve with the incorporation of back-projected images in future studies.

    PR Curve
    
\subsubsection{Justification for Simulations} \label{simulation_justification}


    \paragraph{Thermal} Watertight argument for needing simulations.
    
    We have shown that even on a diverse dataset the CNN performs well. This means it can extract very general 'landmine' features, and the high precision and recall implies that \textit{given some features can be resolved by the sensor}, the CNN is likely to be able to detect it. We can work with the assumption that provided the peak temperature difference between the ambient soil and the soil directly above a buried landmine, is greater than a certain threshold \textbf{what is this threshold! the noise temp or what? Justify me!} Then the CNN will be able to detect the landmine. This is the \textit{detectability hypothesis}
    
    The photos in the dataset are also taken in one geographical area (REF XX), but the detection system should be robust to the specific location of the demining operation (we would like to detect mines in Afghanistan and Ukraine). 
    
    It is thus important that we can simulate the peak temperature differential caused by the presence of a landmine.
    
    \paragraph{RADAR} The motivation behind the radar computer vision simulations is clear; the publicly available datasets are so limited in their diversity and imperfection that they may not be useful for detecting landmines in real, noisy Bscans. We need to produce some Bscans in simulation, where we have used as few assumptions as possible, and try to replicate some of the noise generating processes/feature, so that we may check if our pretrained CNN is robust to real-world noises. If it is, we may say that a CNN trained on real experimental data (a diverse, noisy dataset like that of the thermal sensor) would outperform this model significantly, and thus would perform very well.


