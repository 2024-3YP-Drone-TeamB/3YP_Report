Landmine contamination poses a significant global threat, demanding safer and more efficient detection and clearance methods than traditional manual techniques. This report details the design, simulation, and analysis of a novel multiagent autonomous drone system for landmine detection. The system employs a layered sensing strategy, leveraging the strengths of thermal imaging for rapid wide-area surveying and Ground Penetrating Radar (GPR) for high-confidence confirmation of suspected targets.

Mission planning utilizes Boustrophedon Cellular Decomposition (BCD) for initial thermal Coverage Path Planning (CPP) and a Travelling Salesman Problem with Obstacles (TSP-O) approach, solved via visibility graphs, for targeted GPR scanning. Computer vision models based on the YOLOv11 architecture are implemented for processing thermal and radar image data to identify potential landmine signatures. A key innovation is the physics-informed sensor fusion system using an Adaptive Neuro-Fuzzy Inference System (ANFIS), which integrates environmental context (e.g., soil moisture, wind speed) to improve detection robustness and accuracy beyond traditional methods like Naive Bayes.

The system incorporates custom hardware, including a bespoke Battery Management System (BMS) with Li-ion cells, optimized cell switching, and thermal management for extended endurance and reliability. Robust intra-UAV (CAN bus) and extra-UAV (LoRaWAN) communication architectures are designed for modularity and safety. Control strategies, primarily cascade PID, are tuned and validated through simulation, incorporating wind mitigation techniques using machine learning classifiers trained on sensor data. A dedicated Return-to-Safety (RTS) system using pre-computed cost maps ensures safe landing in hazardous conditions or upon fault detection.

Simulations demonstrate the system's feasibility, validating thermal and radar detectability under challenging conditions (e.g., Afghanistan environment, heterogeneous soil), quantifying sensor fusion performance bounds, and confirming multi-agent coordination and control stability. Economic analysis indicates substantial potential operational cost and time savings compared to manual demining, driven by the system's efficiency and improved precision. This multi-faceted approach presents a promising technological advancement for addressing the humanitarian challenge of landmine clearance.