\subsection{Overview of Fusion} \label{fusion_overview}

    \noindent The late stage fusion approach was introduced in Section \ref{compvis_intro}. It is justified on the grounds of simplicity; by processing each sensor's data through individual YOLO models before merging, the network architectures are simpler. The networks can be retrained in parallel, and there are no single points of failure. 
    


    \noindent In this section, the outputs from the individual YOLO models are interpreted as probability maps, where the confidence value assigned by the YOLO model is equal to the probability of finding a landmine at that location, according to each sensor. The fusion algorithm combines the two probability maps into a third. This is done using the sensor data and environmental contextual data, in a deep learning framework, as discussed in Section \ref{compvis_intro}.

    Fusing the sensor outputs before the CNN is problematic due to irregular input sizes and the need for computationally intensive mosaicking algorithms that may introduce errors. Instead, the accurate position, pose and field of view metadata  is used to \textbf{project each YOLO output onto a common geographical grid.} Given that these probability maps are sparse (most of the probability map is zero, with occasional spikes over landmines), the projections are computationally efficient.

    \subsubsection{The Precision-Recall Trade-off } \label{lossmatrix}

        Recall should be prioritized over precision, as a false negative in landmine detection is more dangerous than a false positive. However, no system can ensure 100\% recall, and so a mined region can never be 100\% safe. Therefore, reducing recall slightly in exchange  for a significant gain in precision can be advantageous, because the reduced recall has negligible operational impact, whilst the increased precision is associated with significant cost reduction, as demonstrated in Section \ref{financing}.